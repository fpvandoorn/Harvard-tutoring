\documentclass[10pt]{article}
\usepackage[utf8]{inputenc}
\usepackage{amsmath,amsfonts,amsthm,amssymb}%fancyhdr
\usepackage{hyperref}
\usepackage{geometry}
%\usepackage[top=1.8in, bottom=1.4in, left=1.4in, right=1.4in]{geometry}
%\usepackage{diagrams}

%\sloppy
\setcounter{secnumdepth}{0}
\setlength{\parindent}{0pt}
\setlength{\parskip}{\baselineskip}
\renewcommand{\arraystretch}{0}
\setlength{\itemsep}{0pt}

\begin{document}
\mbox{}\vspace*{-15mm}
\begin{center}
{\large\bf Syllabus Theorem Proving in Lean}

{\bf Instructors: Floris van Doorn, \url{fpvdoorn@gmail.com}} and {\bf Reid Barton, \url{rwbarton@gmail.com }}
\end{center}

\textbf{About the course}:
This will be an 8-week course in February--April where we will give an hands-on introduction to the proof assistant Lean. We will start with an overview and the basics of Lean and \textsf{mathlib} (Lean's mathematical library), and then dive into using Lean in various areas of mathematics.

Every week we will have a 1-hour group meeting, 1-hour individual sessions and we expect all students to spend 3-5 hours on their own.

The group meetings will take place on Zoom with meeting ID 998 3503 8916 (\url{https://pitt.zoom.us/j/99835038916}) and passcode \textsf{mathlib}. Time TBD. Before each session we will assign some reading and/or video to watch about a particular topic, and then teach and discuss about that topic during the meeting. You are encouraged to ask questions!

Outside the meetings we expect all students to prepare for the group meetings, and either work on exercises or an individual project.

Each student will schedule a weekly meeting with one of the instructors, either individually or with one other student. In these meetings we will discuss the exercises and project, and we can do some ``pair programming'' where the student shares their screen, and the instruction can give immediate feedback.

\textbf{Online materials}:
Learning recourses can be found on this page: \url{https://leanprover-community.github.io/learn.html}

In particular:
\begin{itemize}
\item The online ``game'' \href{http://wwwf.imperial.ac.uk/~buzzard/xena/natural_number_game/}{Natural Number Game}
\item The book \href{https://leanprover.github.io/theorem_proving_in_lean/}{Theorem Proving in Lean}
\item The \href{https://www.youtube.com/playlist?list=PLlF-CfQhukNlxexiNJErGJd2dte_J1t1N}{Recorded lectures} and \href{https://leanprover-community.github.io/lftcm2020/exercises.html}{Exercises} of the conference \emph{Lean for the Curious Mathematician 2020}.
\item A tactics \href{https://leanprover-community.github.io//img/lean-tactics.pdf}{Cheatsheet}
\end{itemize}

\textbf{Projects}:
We encourage each student to work on an individual project, once they have learned the basics of Lean. A good project is something that is within reach of \textsf{mathlib}. To see what is in \textsf{mathlib}, a good place to look is the \href{https://leanprover-community.github.io/mathlib-overview.html}{library overview} or the \href{https://leanprover-community.github.io/undergrad.html}{undergraduate topics in \textsf{mathlib}}.

A good place to look for topics is the \href{https://leanprover-community.github.io/undergrad_todo.html}{list of undergraduate topics not yet in \textsf{mathlib}}. Or you can start formalizing the basics of an area in math that you are interested in. Please note that many graduate topics in mathematics do not yet exist in \textsf{mathlib}.


\end{document}
